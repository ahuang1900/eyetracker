\documentclass[%
%handout, % prints handouts (=no animations, for printed version)
%mathserif
%xcolor=pst,
14pt
%fleqn
]{beamer}

\usepackage{beamerthemedefault}

\useoutertheme[subsection=false]{smoothbars}
\useinnertheme[shadow=true]{rounded}

\setbeamercolor{block title}{fg=black,bg=gray}
\setbeamercolor{block title alerted}{use=alerted text,fg=black,bg=alerted text.fg!75!bg}
\setbeamercolor{block title example}{use=example text,fg=black,bg=example text.fg!75!bg}
\setbeamercolor{block body}{parent=normal text,use=block title,bg=block title.bg!25!bg}
\setbeamercolor{block body alerted}{parent=normal text,use=block title alerted,bg=block title alerted.bg!25!bg}
\setbeamercolor{block body example}{parent=normal text,use=block title example,bg=block title example.bg!25!bg}

%% for including video/audio
%\usepackage{multimedia} %needs hyperref-package
%\usepackage{movie15}

%\usepackage{xmpmulti}

%% aktive Referenzen
%\usepackage{hyperref}

%\usepackage{ngerman}			% language set to new-german
\usepackage[english]{babel}			% language set to new-german
\usepackage[utf8]{inputenc} 	% coding of german special characters
% \usepackage{ae,aecompl}
\usepackage{amsmath,amssymb,amstext} 	% support for mathematics

% \usepackage{listings}		% include programming code
%\usepackage{amsfonts}		% blackboard fonts: $\mathbb{N,Z,R,C,...}$
%\usepackage{latexsym}
%\usepackage{textcomp}
%\usepackage{mathptmx,courier}
			% \textdegree \textcelsius \textperthousand
			% \copyright \texttrademark \textregistered
			% \textmu (non-italic mu)
%\usepackage{geometry}	% change paper dimension an margins
%    \geometry{verbose,paperwidth=128mm,paperheight=90mm}
%    \geometry{tmargin=0mm,bmargin=0mm,lmargin=10mm,rmargin=10mm}
\usepackage[]{graphicx}  % \includegraphics[options]{file.eps}
\usepackage{subfigure}
			% options = scale, width, totalheight, height, depth,
			% angle = deg, origin = {l c r}{top, Baseline, bottom}
			% \scalebox{h-scale}[v-scale]{object}
			% \rotatebox[x=xdim,y=ydim]{angleCCW}[object}
%\usepackage{tabularx}	% \begin{tabular}{...X...} stretches column
%\usepackage{multirow}
%\usepackage{floatflt}
%\usepackage{hhline}
%\usepackage{colortbl}
\usepackage{array}
\usepackage{setspace}

\usepackage{tikz}
\usetikzlibrary{arrows,positioning}

%\usepackage[thinspace,thinqspace,squaren,textstyle]{SIunits}

\definecolor{tug}{rgb}{0.96862,0.14509,0.27450}

\setbeamertemplate{headline}[text line]{
	\begin{beamercolorbox}[wd=\paperwidth,ht=8ex,dp=4ex]{}
		\insertnavigation{0.85\paperwidth} 
		\raisebox{-10pt}{\includegraphics[width=15mm]{tuglogo}}\vskip2pt
		\hskip-1pt\rule{\paperwidth}{0.3pt}
	\end{beamercolorbox}
}

\setbeamertemplate{navigation symbols}{}

\definecolor{gray}{rgb}{0.8,0.8,0.8}
\setbeamercolor{footline}{fg=black,bg=gray}

% Fußzeile mit Autor, Titel und Foliennummer / Gesamtfolienzahl
\setbeamertemplate{footline}[text line]{
	\hskip-1pt
	\begin{beamercolorbox}[wd=\paperwidth]{footline}
			\rule{\paperwidth}{0.3pt}
			\colorbox{tug}{\rule{3pt}{0pt}\rule{0pt}{3pt}}
			\textbf{\rule{0pt}{5pt}\insertshortauthor\hfill\insertshortinstitute\hfill%
					\insertshorttitle\rule{1em}{0pt}}
			\rule{\paperwidth}{0.3pt}
	\end{beamercolorbox}
	\begin{beamercolorbox}[wd=\paperwidth,ht=2ex,dp=2ex]{white}
	\end{beamercolorbox}
}%
\usepackage{xspace}

%% Titelblatt-Einstellungen
%\institute[TU Graz]{Institute of Medical Engineering, TU Graz }
\title{EyeTracking with two cameras}
\subtitle{Image Processing and Pattern Recognition Project}
\author{Stefan Kroboth, Christoph Aigner}
\date{January 23, 2012}		% wenn ein anderes als das heutige Datum eingesetzt werden soll

% Subject und Keywords für PDF-Datei
\subject{EyeTracking with two cameras}
\keywords{EyeTracking, Eye, Pupil, Tracking}

\titlegraphic{\includegraphics[width=20mm]{tuglogo}}

%%%%%%%%%%%%%%%%%%%%%%%%%%%%%%%%%%%%%%%%%%%%%%%%%%%%%%%%%%%%%%%%%%%%%%%%%%%%
\begin{document}
%%%%%%%%%%%%%%%%%%%%%%%%%%%%%%%%%%%%%%%%%%%%%%%%%%%%%%%%%%%%%%%%%%%%%%%%%%%%

\begin{frame}[plain]
  \frametitle{}
  \titlepage % erzeugt Titelseite
\end{frame}



\begin{frame}
  \frametitle{Contents}
  \begin{spacing}{0.7}
        \tableofcontents[hideallsubsections %
                        % ,pausesections
                        ] % erzeugt Inhaltsverzeichnis
                      \end{spacing}
\end{frame}

%%%%%%%%%%%%%%%%%%%%%%%%%%%%%%%%%%%%%%%%%%%%%%%%%%%%%%%%%%%%%%%%%%%%%%%%%%%%
\section{Introduction}
%%%%%%%%%%%%%%%%%%%%%%%%%%%%%%%%%%%%%%%%%%%%%%%%%%%%%%%%%%%%%%%%%%%%%%%%%%%%
\subsection{Introduction}
\begin{frame}
	\frametitle{Introduction}
  \begin{itemize}
    \item Goal: Map eye movement onto screen independent of head position
    \item two cameras
      \begin{itemize}
        \item eye camera $\rightarrow$ track eye movement
        \item head camera $\rightarrow$ track head movement
      \end{itemize}
  \end{itemize}
\end{frame}

\section{Methods}
\subsection{Hardware}
\begin{frame}
	\frametitle{Hardware}
  %\framesubtitle{Reconstruction}
  \begin{itemize}
    \item lots of images
  \end{itemize}
\end{frame}

\subsection{Software}
\begin{frame}
	\frametitle{Software}
  \framesubtitle{Idea}
  \begin{itemize}
    \item Scheme
  \end{itemize}
\end{frame}

\subsection{Basic idea}
\subsubsection{The math}
\begin{frame}
	\frametitle{Basic idea}
  \framesubtitle{The ``math''}
  \begin{block}{Basic idea}
    \begin{equation*}
      x_\mathrm{screen} = H x_\mathrm{eye}
    \end{equation*}
  \end{block}\pause
  \begin{block}{Calibration}
    \begin{equation*}
      x_\mathrm{screen} = H_\mathrm{calib} (H_\mathrm{calib-head} x_\mathrm{eye})
    \end{equation*}
  \end{block}\pause
  \begin{block}{Calibrated mapping}
    \begin{equation*}
      H_\mathrm{calib} = H_\mathrm{eye} H_\mathrm{calib-head} ^{-1}
    \end{equation*}
  \end{block}
\end{frame}

\begin{frame}
	\frametitle{Basic idea}
  \framesubtitle{The ``math''}
  \begin{block}{Mapping without head movement}
    \begin{equation*}
      x_\mathrm{screen} = H_\mathrm{eye} H_\mathrm{calib-head} ^{-1} H_\mathrm{calib-head} x_\mathrm{eye}
    \end{equation*}
    \begin{equation*}
      x_\mathrm{screen} = H_\mathrm{eye}  x_\mathrm{eye}
    \end{equation*}
  \end{block}\pause
  \begin{block}{Mapping with head movement}
    \begin{equation*}
      x_\mathrm{screen} = H_\mathrm{eye} H_\mathrm{calib-head} ^{-1} H_\mathrm{head} x_\mathrm{eye}
    \end{equation*}
    \begin{equation*}
      H_\mathrm{rel-head} = H_\mathrm{calib-head}^{-1} H_\mathrm{head}
    \end{equation*}
  \end{block}
\end{frame}

\subsubsection{Head Tracking}
\begin{frame}
	\frametitle{Software}
  \framesubtitle{Head Tracking}
  \begin{itemize}
    \item Scheme
  \end{itemize}
\end{frame}
\subsubsection{Eye Tracking}
\begin{frame}
	\frametitle{Software}
  \framesubtitle{Eye Tracking}
  \begin{itemize}
    \item Scheme
  \end{itemize}
\end{frame}


\section{Results}
\subsection{Video}
\begin{frame}
	\frametitle{Results}
  \framesubtitle{Video}
\end{frame}

\section{Discussion}
\subsection{Problems}
\begin{frame}
	\frametitle{Discussion}
  \framesubtitle{Problems}
  \begin{itemize}
    \item bad hardware
      \begin{itemize}
        \item removed IR cutoff filter $\rightarrow$ flickering
        \item bad resolution of eye camera
        \item IR cutoff filter head camera $\rightarrow$ high integration time, harder to track IR LEDs
      \end{itemize}
    \item little inaccuracies have big effect
      \begin{itemize}
        \item where is the eye frame within the markers?
        \item what is the relation of the eye frame to the markers?
      \end{itemize}
  \end{itemize}
\end{frame}
\subsection{Conclusion}
\begin{frame}
	\frametitle{Discussion}
  \framesubtitle{Conclusion}
  \begin{itemize}
    \item works in principle
    \item perfect eye tracking is crutial
    \item knowledge of the geometries is crutial as well
  \end{itemize}
\end{frame}


\end{document}
%%%%%%%%%%%%%%%%%%%%%%%%%%%%%%%%%%%%%%%%%%%%%%%%%%%%%%%%%%%%%%%%%%%%%%%%%%%%

%% EOF
